\section{Jan 17, 2025}
\subsection{Base Kets and Matrix Representations}
\begin{frame}{Eigenkets of an observation}
	\begin{alertblock}{Theorem 1}
		The eigenvalues of a Hermitian operator $A$ are real; the eigenkets of $A$ corresponding to different eigenvalues are orthogonal.
	\end{alertblock}
	Proof:

	\begin{gather}
		A\ket{a'} = a'\ket{a'} \label{eq:1}\\
		\bra{a'' }A = a''^* \bra{a''} \label{eq:2}\\
		\bra{a''} \cdot \ref{eq:1} - \ref{eq:2}\cdot \ket{a'} \to (a'-a'') \ip{a''}{a'} = 0
	\end{gather}
	If $a'=a''$, then $a'=a''^*=a'^*$ since $\ip{a'}{a'} = 0$

	$\therefore a'$ is real. If $a'\ne a'', a'-a'' \ne 0 \Longrightarrow \ip{a''}{a'} = 0$

	i.e. $\ket{a''} \& \ket{a'}$ are orthogonal.

\end{frame}
\begin{frame}
	Can normalize $\ket{a'}$ so that $\{\ket{a'}\}$ forms an orthogonal set with $\ip{a''}{a'}= \delta_{a' a''}$.

	The set of eigenkets forms a complete set to span the ket space.
\end{frame}

\subsection{Eigenkets and Base Kets}
\begin{frame}
	\frametitle{Eigenkets and Base Kets}

	\begin{equation}\label{eq:alpha base}
		\ket{\alpha} = \sum_{a'} C_{a'} \ket{a'}
	\end{equation}
	--- expansion of an arbitrary ket in the ket space spanned by the eigenkets of $A$.
	\begin{gather*}
		\bra{a''} (\ref{eq:alpha base}): \quad \ip{a''}{\alpha} = \sum_{a'} C_{a'} \underbrace{\ip{a''}{a'}}_{\delta_{a'' a'}} = C_{a''}\\
		\Rightarrow C_{a'} = \ip{a'}{\alpha}\\
		\therefore \quad \ket{\alpha} = \sum_{a'} \op{a'}\ket{\alpha}\\
		\text{C.f.} \quad \vec{V} = \sum_i \hat e_i (\hat e_i \cdot \vec V) \qq{for a vector} \vec V
	\end{gather*}
\end{frame}
