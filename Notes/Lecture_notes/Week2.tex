\section{Jan 17, 2025}
\subsection{Base Kets and Matrix Representations}
\begin{frame}{Eigenkets of an observation}
	\begin{alertblock}{Theorem 1}
		The eigenvalues of a Hermitian operator $A$ are real; the eigenkets of $A$ corresponding to different eigenvalues are orthogonal.
	\end{alertblock}
	Proof:

	\begin{gather}
		A\ket{a'} = a'\ket{a'} \label{eq:1}\\
		\bra{a'' }A = a''^* \bra{a''} \label{eq:2}\\
		\bra{a''} \cdot \ref{eq:1} - \ref{eq:2}\cdot \ket{a'} \to (a'-a'') \ip{a''}{a'} = 0
	\end{gather}
	If $a'=a''$, then $a'=a''^*=a'^*$ since $\ip{a'}{a'} = 0$

	$\therefore a'$ is real. If $a'\ne a'', a'-a'' \ne 0 \Longrightarrow \ip{a''}{a'} = 0$

	i.e. $\ket{a''} \& \ket{a'}$ are orthogonal.

\end{frame}
\begin{frame}
	Can normalize $\ket{a'}$ so that $\{\ket{a'}\}$ forms an orthogonal set with $\ip{a''}{a'}= \delta_{a' a''}$.

	The set of eigenkets forms a complete set to span the ket space.
\end{frame}

\subsection{Eigenkets and Base Kets}
\begin{frame}
	\frametitle{Eigenkets and Base Kets}

	\begin{equation}\label{eq:alpha base}
		\ket{\alpha} = \sum_{a'} C_{a'} \ket{a'}
	\end{equation}
	--- expansion of an arbitrary ket in the ket space spanned by the eigenkets of $A$.
	\begin{gather*}
		\bra{a''} (\ref{eq:alpha base}): \quad \ip{a''}{\alpha} = \sum_{a'} C_{a'} \underbrace{\ip{a''}{a'}}_{\delta_{a'' a'}} = C_{a''}\\
		\Rightarrow C_{a'} = \ip{a'}{\alpha}\\
		\therefore \quad \ket{\alpha} = \sum_{a'} \op{a'}\ket{\alpha}\\
		\text{C.f.} \quad \vec{V} = \sum_i \hat e_i (\hat e_i \cdot \vec V) \qq{for a vector} \vec V
	\end{gather*}
\end{frame}


\begin{frame}
	\frametitle{Completeness relation or closure relation}
	$$
		\mathbb{1} = \sum_{a'} \op{a'}
	$$
	eg.
	\begin{columns}
		\column{.5\textwidth}
		\begin{gather*}
			\ip{\alpha} = \bra{\alpha} \cdot \qty(\sum_{a'}\op{a'}) \cdot \ket{\alpha}	\\
			= \sum_{a'} \ip{\alpha}{a'}\ip{a'}{\alpha}\\
			= \sum_{a'} \ip{a'}{\alpha}^*\ip{a'}{\alpha}\\
			= \sum_{a'} \qty|\ip{a'}{\alpha}|^2
		\end{gather*}
		\column{.5\textwidth}
		also if $\ket{\alpha}$ is normalized then $\ip{\alpha}  = 1$
		\begin{align*}
			\Rightarrow & \sum_{a'} \qty|\ip{a'}{\alpha}|^2 \\
			            & = \sum_{a'} |C_{a'}|^2            \\
			            & = 1
		\end{align*}
	\end{columns}
\end{frame}
\begin{frame}
	\frametitle{Projection operator}

	Note: $\op{a'}$ is a projection operator because
	\[
		\qty(\op{a'})\cdot \ket{\alpha} = \ket{a'} \ip{a'}{\alpha} = C_{a'} \ket{a'}
	\]
	c.f. Dyadic ooperator in 3D Euclidean space
	\[\hat e_i \hat e_j \qquad (\hat x \hat x) \cdot \bar V = \hat x (\hat x \cdot \bar V)\]
\end{frame}
\section{Matrix Representations}
\begin{frame}
	\frametitle{Matrix representation}
	A ket can be represented as a column vector.\\
	An operator can be represented as a matrix.
	\[
		\ket \alpha = \sum_{a'} \ket{a'} \ip{a'}{\alpha}
	\]
	where $a'$ is a base ket.

	If we use a column vector to represent $\ket{a'}$ it will be the following. Denote
	\begin{gather*}
		\qty{a'} = \qty{a^{(1)},a^{(2)},\dots}\\
		\ket{a^{(1)}} = \mqty[1 \\ 0 \\ 0 \\ \vdots] \quad \ket{a^{(2)}} = \mqty[0 \\ 1 \\ 0 \\ \vdots]
	\end{gather*}

\end{frame}
\begin{frame}
	\frametitle{Matrix Representations}

	So
	\[
		\ket{\alpha} = \ket{a^{(1)}} = \mqty[\ip{a^{(1)}}{\alpha} \\ \ip{a^{(2)}}{\alpha} \\ \vdots] = \mqty[C_{a^{(1)}}\\C_{a^{(2)}}\\ \vdots]
	\]
	Operator $X$ in matrix form
	\[
		X = \mathbb{1} \cdot X \cdot \mathbb{1} = \sum_{a'} \op{a''} X\sum_{a'} \op{a'}  = \sum_{a'' a'} \op{a''} X \op{a'}
	\]
	\[
		X = \mqty[\ev{A}{a^{(1)}} \\ 1]
	\]
\end{frame}
