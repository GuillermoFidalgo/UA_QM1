\section{Chapter 1}
\subsection[Base Kets]{Base Kets and Matrix Representations}
\begin{frame}{Eigenkets of an observation}
	\begin{theorem}[1]
		The eigenvalues of a Hermitian operator $A$ are real; the eigenkets of $A$ corresponding to different eigenvalues are orthogonal.
	\end{theorem}
	Proof:

	\begin{gather}
		A\ket{a'} = a'\ket{a'} \label{eq:1}\\
		\bra{a'' }A = a''^* \bra{a''} \label{eq:2}\\
		\bra{a''} \cdot \ref{eq:1} - \ref{eq:2}\cdot \ket{a'} \to (a'-a'') \ip{a''}{a'} = 0
	\end{gather}
	If $a'=a''$, then $a'=a''^*=a'^*$ since $\ip{a'}{a'} = 0 \to $ a is real.

	If $a'\ne a'', a'-a'' \ne 0 \Longrightarrow \ip{a''}{a'} = 0 \to \ket{a''} \& \ket{a'}$ are orthogonal.

	\vfill

	Can normalize $\ket{a'}$ so that $\{\ket{a'}\}$ forms an orthogonal set with $\ip{a''}{a'}= \delta_{a' a''}$.
	The set of eigenkets forms a complete set to span the ket space.
\end{frame}

\subsubsection{Eigenkets as Base Kets}
\begin{frame}
	\frametitle{Eigenkets as Base Kets}

	\begin{equation}\label{eq:alpha base}
		\ket{\alpha} = \sum_{a'} C_{a'} \ket{a'}
	\end{equation}
	Expansion of an arbitrary ket in the ket space spanned by the eigenkets of $A$.
	\begin{gather*}
		\bra{a''} (\ref{eq:alpha base}): \quad \ip{a''}{\alpha} = \sum_{a'} C_{a'} \underbrace{\ip{a''}{a'}}_{\delta_{a'' a'}} = C_{a''}\\
		\Rightarrow C_{a'} = \ip{a'}{\alpha}\\
		\therefore \quad \ket{\alpha} = \sum_{a'} \op{a'}\ket{\alpha}\\
		\text{C.f.} \quad \vec{V} = \sum_i \hat e_i (\hat e_i \cdot \vec V) \qq{for a vector} \vec V
	\end{gather*}
\end{frame}


\begin{frame}
	\frametitle{Completeness relation or closure relation}
	$$
		\mathbb{1} = \sum_{a'} \op{a'}
	$$
	eg.
	\begin{columns}
		\column{.4\textwidth}
		\begin{gather*}
			\ip{\alpha} = \bra{\alpha} \cdot \qty(\sum_{a'}\op{a'}) \cdot \ket{\alpha}	\\
			= \sum_{a'} \ip{\alpha}{a'}\ip{a'}{\alpha}\\
			= \sum_{a'} \ip{a'}{\alpha}^*\ip{a'}{\alpha}\\
			= \sum_{a'} \qty|\ip{a'}{\alpha}|^2
		\end{gather*}
		\column{.02\textwidth}

		\rule{.1mm}{0.45\textheight}

		\column{.4\textwidth}
		also if $\ket{\alpha}$ is normalized then $\ip{\alpha}  = 1$
		\begin{align*}
			\Rightarrow & \sum_{a'} \qty|\ip{a'}{\alpha}|^2 \\
			            & = \sum_{a'} |C_{a'}|^2            \\
			            & = 1
		\end{align*}
	\end{columns}
\end{frame}
\begin{frame}
	\frametitle{Projection operator}

	Note: $\op{a'}$ is a projection operator because
	\[
		\qty(\op{a'})\cdot \ket{\alpha} = \ket{a'} \ip{a'}{\alpha} = C_{a'} \ket{a'}
	\]
	c.f. Dyadic ooperator in 3D Euclidean space
	\[\hat e_i \hat e_j \qquad (\hat x \hat x) \cdot \bar V = \hat x (\hat x \cdot \bar V)\]
\end{frame}
\subsubsection{Matrix Representations}
\begin{frame}
	\frametitle{Matrix Representations}
	A ket can be represented as a column vector.\\
	An operator can be represented as a matrix.
	\[
		\ket \alpha = \sum_{a'} \ket{a'} \ip{a'}{\alpha}
	\]
	where $a'$ is a base ket.

	If we use a column vector to represent $\ket{a'}$ it will be the following. Denote
	\begin{gather*}
		\qty{a'} = \qty{a^{(1)},a^{(2)},\dots}\\
		\ket{a^{(1)}} = \mqty[1 \\ 0 \\ 0 \\ \vdots] \quad \ket{a^{(2)}} = \mqty[0 \\ 1 \\ 0 \\ \vdots]
	\end{gather*}

\end{frame}
\begin{frame}
	\frametitle{Matrix Representations}

	So
	\[
		\ket{\alpha} = \ket{a^{(1)}} = \mqty[\ip{a^{(1)}}{\alpha} \\ \ip{a^{(2)}}{\alpha} \\ \vdots] = \mqty[C_{a^{(1)}}\\C_{a^{(2)}}\\ \vdots]
	\]
	Operator $X$ in matrix form
	\[
		X = \mathbb{1} \cdot X \cdot \mathbb{1} = \sum_{a'} \op{a''} X\sum_{a'} \op{a'}  = \sum_{a'' a'} \op{a''} X \op{a'}
	\]
	\[
		X = \mqty[
			\ev{X}{a^{(1)}} & \mel{a^{(1)}}{X}{a^{(2)}} & \cdots \\
			\mel{a^{(2)}}{X}{a^{(1)}} & \ev{X}{a^{(2)}} & \cdots \\
			\vdots & \vdots & \ddots
		]
	\]
	is a square matrix, note:

	\[
		\mel{a'' }{X}{a'} = \mel{a'}{X^\dag}{a''}^* \Rightarrow \mel{a'}{X^\dag}{a''} = \mel{a''}{X}{a'}^*
	\]
	Hermitian adjoint corresponds to taking the transpose and complex conjugate of the matrix form of the operator.
\end{frame}

\begin{frame}
	\frametitle{Matrix Representations}

	\begin{gather*}
		Z = XY\\
		\mel{a''}{Z}{a'} = \mel{a''}{XY}{a'} \\
		= \sum_{a'''} \mel{a''}{X}{a'''} \mel{a'''}{Y}{a'}\\
		\ket{\gamma} = X \ket{\alpha}\\
		\ip{a'}{\gamma} = \mel{a'}{X}{\alpha} \Rightarrow \sum_{a''} \mel{a'}{X}{a''}\ip{a''}{\alpha} \\
		\mqty[\ip{a^{(1)}}{\gamma} \\ \ip{a^{(2)}}{\gamma} \\ \vdots] = \mqty[
			\ev{X}{a^{(1)}} & \mel{a^{(1)}}{X}{a^{(2)}} & \cdots \\
			\mel{a^{(2)}}{X}{a^{(1)}} & \ev{X}{a^{(2)}} & \cdots \\
			\vdots & \vdots & \ddots
		] \mqty[\ip{a^{(1)}}{\alpha} \\ \ip{a^{(2)}}{\alpha} \\ \vdots]
	\end{gather*}
	same for bras
\end{frame}

\begin{frame}{Matrix Representations}
	\begin{gather*}
		\bra{\gamma} = \bra{\alpha} X \\
		\ip{\gamma}{a'} = \mel{\alpha}{X}{a'} = \sum_{a''} \ip{\alpha}{a''}\mel{a'' }{X}{a'}\\
		= \mqty[\ip{\gamma}{a^{(1)}} & \ip{\gamma}{a^{(2)}} & \cdots]  \\
		= \mqty[\ip{\alpha}{a^{(1)}} & \ip{\alpha}{a^{(2)}} & \cdots]
		\mqty[
			\ev{X}{a^{(1)}} & \mel{a^{(1)}}{X}{a^{(2)}} & \cdots \\
			\mel{a^{(2)}}{X}{a^{(1)}} & \ev{X}{a^{(2)}} & \cdots \\
			\vdots & \vdots & \ddots
		]
	\end{gather*}

\end{frame}

\begin{frame}{Matrix Representations}
	For inner products
	\begin{gather*}
		\ip{\beta}{\alpha} = \sum_{a'} \ip{\beta}{a'}\ip{a'}{\alpha} \\
		= \mqty[ \ip{\beta}{a^{(1)}} & \ip{\beta}{a^{(2)}} & \cdots]
		\mqty[ \ip{a^{(1)}}{\alpha} \\
			\ip{a^{(2)}}{\alpha} \\
			\vdots]
		\\
	\end{gather*}
	For outer products
	\begin{gather*}
		\op{\beta}{\alpha} = \qty(\sum_{a''} \op{a''}) \ket\beta \bra{\alpha} \qty(\sum_{a'} \op{a'})=\sum_{a' a''} \op{a''}\op{\beta}{\alpha} \op{a'} \\
		= \ip{a'''}{\beta} \ip{\alpha}{a'''} = \mqty[
			\ip{a^{(1)}}{\beta}\ip{a^{(1)}}{\alpha}^* & \ip{a^{(1)}}{\beta}\ip{a^{(2)}}{\alpha}^* & \cdots \\
			\ip{a^{(2)}}{\beta}\ip{a^{(1)}}{\alpha}^* & \ip{a^{(2)}}{\beta}\ip{a^{(1)}}{\alpha}^* & \cdots \\
			\vdots & \vdots & \ddots
		]
	\end{gather*}
\end{frame}

\begin{frame}{Matrix Representations}
	\begin{itemize}
		\item If $\{\ket{a'}\}$ is the set of eigenkets of the operator $A = \mathbb{1}\cdot A \cdot \mathbb{1}$
	\end{itemize}
	\begin{gather*}
		\sum_{a'' a'} \op{a''} \underbrace{A\ket{a'}}_{a'\ket{a'}}\bra{a'} = \sum_{a' a''} a'\ket{a'}\underbrace{\ip{a''}{a'}}_{\delta_{a'' a'}} \ket{a'} = \sum_{a'} a'\op{a'}\\
		\mqty[\dmat{ a^{(1)},a^{(2)},\ddots }]
	\end{gather*}
\end{frame}

\subsubsection{Example: Spin 1/2 systems}
\begin{frame}{Example: Spin 1/2 systems}
	Base kets : $ \ket{S_z;\pm} = \ket{\pm}$\\
	Completeness: $\mathbb{1} = \op{+} + \op{-}$
	\begin{gather*}
		\text{Eigenvalue eqn}: S_z \ket{\pm} = \pm \frac\hbar2 \ket{\pm}\\
		S_z = \frac\hbar2 \op{+} - \frac{\hbar}{2}\op{-} = \frac{\hbar}{2} \qty(\op{+} - \op{-})
	\end{gather*}
\end{frame}

\begin{frame}{Example: Spin 1/2 systems}

	Now define

	\[
		\begin{split}
			S_+ = \hbar \op{+}{-} \\
			S_- = \hbar \op{-}{+}
		\end{split}\Biggr\} \text{Ladder Operators}
	\]

	Note:
	\[
		\begin{split}
			S_+ \ket{-} = \hbar \ket + \overbrace{\ip{-}}^1 = \hbar\ket + \\
			S_+ \ket{+} = \hbar \ket + \underbrace{\ip{-}{+}}_0 =  0      \\
		\end{split} \Biggr\} \text{Raising Operator}
	\]
	%
	\[
		\begin{split}
			S_- \ket{+} = \hbar \ket - \overbrace{\ip{+}}^1 = \hbar\ket - \\
			S_- \ket{-} = \hbar \ket - \underbrace{\ip{+}{-}}_0 =  0      \\
		\end{split} \Biggr\} \text{Lowering Operator}
	\]

\end{frame}


\begin{frame}{Matrix Representations}{Spin 1/2 systems}
	$$\ket{+} = \mqty[1 \\ 0] \qquad \ket- = \mqty[0 \\ 1]$$
	\begin{gather*}
		S_z \doteq \frac{\hbar}{2} \pqty{\mqty[1\\0] \mqty[1 & 0] - \mqty[0 \\ 1] \mqty[0 & 1]} = \frac{\hbar}{2} \mqty[\pmat{3}]\\
		S_+ = \hbar \op{+}{-} = \hbar \mqty[1 \\ 0 ] \mqty[0 & 1] = \hbar \mqty[0 & 1\\0 & 0]\\
		S_- = \hbar \mqty[0&0\\1&0]
	\end{gather*}
\end{frame}


\subsection[Observables]{Measurements, Observables and the Uncertainty Relations}

\begin{frame}{Measurements, Observables and the Uncertainty Relations}{Measurements}
	\begin{itemize}
		\item Observable A \\
		      Before measurement, the system is in a state $\ket \alpha$
		      \[
			      \ket{\alpha} = \sum_{a'} \ket{a'} \ip{a'}{\alpha}
		      \]
		      where $\{\ket{a'}\}$ is the set of eigenkets for $A$. Under measuerment the system is thrown into an eigenstate
		      \[ \ket{\alpha} \xrightarrow{\text{ \tiny A measurement}} \ket{a'}\]
		      If the state before measurement is an eigenstate of the observables, say $\ket{a'}$
		      \[ \ket{a'} \xrightarrow{\text{\tiny A measurement}} \ket{a'}\]
	\end{itemize}
\end{frame}

\begin{frame}
	\begin{itemize}
		\item Probability for jumping into some particular state $\ket{a'}$ is $|\ip{a'}{\alpha}|^2$ where $\ket{\alpha}$ is normalized.
		      In the particular case when $\ket{\alpha} = \ket{a'}$ before measurement, then probability of measuring $A$ to be $a'$ will be 1 since
		      \[\qty|\ip{a'}|^2 = 1\]
		      Probability of measuring $A$ to be $a''$ (where $a'' \neq a'$) will be 0 since $\qty|\ip{a''}{a'}|^2 = 0$ due to orthogonality.
		\item Expectation value of $A$ when the system is in state $\ket \alpha$. \[\ev{A} =\ev{A}{\alpha} \]
		      means the average measured value
		      \begin{gather*}
			      \ev{A} = \sum_{a''}\sum_{a'} \ip{\alpha}{a''}\mel{a''}{A}{a'}\ip{a'}{\alpha}\\
			      = \sum_{a''}\sum_{a'} a'\ip{\alpha}{a''} \underbrace{\ip{a''}{a'}}_{\delta_{a''a'}} \ip{a'}{\alpha}\\
			      = \sum_{a'} a' \ip{\alpha}{a'}\ip{a'}{\alpha} = \sum_{a'} a' \ip{a'}{\alpha} \ip{a'}{\alpha}^* = \sum_{a'} a'|\ip{a'}{\alpha}|^2
		      \end{gather*}
	\end{itemize}
\end{frame}

\subsubsection{Compatible Observables}
\begin{frame}{Compatible Observables}
	\begin{itemize}
		\item Compatible observables if $\comm{A}{B} = 0$, otherwise they are called incompatible. e.g.
		      \begin{gather*}
			      [S^2,S_z] = 0 \Rightarrow S^2,S_z \; \text{are compatible}\\
			      \comm{S_x}{S_z} \neq 0 \Rightarrow S_x,S_z \; \text{are incompatible}
		      \end{gather*}
		\item Suppose there are two (or more) linearly independent eigenkets of $A$ having the same eigenvalue, then the eigenvalues of the two eigenkets are said to be degenerate. e.g. $\left\{ a^{(1)}, a^{(2)},\dots,a^{(k)}\right\}$ correspond to $\left\{\ket{a^{(1)}},\ket{a^{(2)}},\dots,\ket{a^{(k)}}	\right\}$
		      If $a^{(1)} = a^{(2)} = a, \ket{a^{(1)}} \& \ket{a^{(2)}}$ are degenerate states $\Rightarrow \{a,a,a^{(3)} ,\dots\}$. How to distinguish them?
	\end{itemize}
\end{frame}

\begin{frame}{Compatible Observables}
	\begin{theorem}[2]
		Suppose $A \ \&\  B$ are compatible observables, and the eigenvalues of $A$ are non-degenerate. Then, the matrix elements $\mel{a''}{B}{a'}$ are all diagonal.
	\end{theorem}

	Proof:
	\begin{align*}
		\mel{a''}{\comm{A}{B}}{a'} & = \mel{a''}{(AB - BA)}{a'}    \\
		                           & = \mel{a''}{(a''B -B a')}{a'} \\
		                           & = (a'' - a') \mel{a''}{B}{a'} \\
		                           & = 0
	\end{align*}
	$\mel{a''}{B}{a'} = 0 $ unless $a'' = a'$
	\[
		\mqty[\ddots\admat{0,\ddots,0}\ddots]
	\]
\end{frame}

\begin{frame}
	If $\comm{A}{B} = 0$ and eigenvalues of $A$ are non-degenerate $ \mel{a''}{B}{a'} $ are diagonal.
	\begin{gather*}
		\mel{a''}{B}{a'}  = \delta_{a'a''} \ev{B}{a'}\\
		B = \mathbb{1}\cdot B \cdot \mathbb{1} = \sum_{a' a''} \ket{a''} \underbrace{\mel{a''}{B}{a'} }_{\delta_{a'a'} \ev{B}{a'}}\bra{a'}\\
		\sum_{a''}\ket{a''}\mel{a''}{B}{a''} \bra{a''}\\
		B\ket{a'} = \sum_{a''} \ket{a''}\mel{a''}{B}{a''}\underbrace{\ip{a''}{a'}}_{\delta_{a'a''}}\\
		= \ket{a'} \underbrace{\mel{a'}{B}{a'}}_{b'} = b'\ket{a'} \; \to \text{eigenvalue eq. for }B.
	\end{gather*}
	$\ket{a'}$ is a simultaneous eigenket of $A$ and $B$ labeled as $\ket{a',b'}$
	\begin{align*}
		A\ket{a',b'} & =a'\ket{a',b'} \\
		B\ket{a',b'} & =b'\ket{a',b'}
	\end{align*}
\end{frame}

\begin{frame}
	If we have a maximal set of mutually commuting observables $A,B,\dots$ with $\comm{A}{B}=\comm{B}{C}=\comm{A}{C}= 0$.\\
	Simultaneous eigenkets are labeled as $\ket{k'} = \ket{a',b',c',\dots}$.
	\begin{itemize}
		\item Orthonormality: $\ip{k''}{k'} = \delta_{k''k'}=\delta_{a''a'}\delta_{b''b'}\delta_{c''c'}$
		\item Completeness: $\sum_{k'}\op{k'} = \sum_a' \sum_b' \sum_c' \dots \op{a',b',c',\dots} = \mathbb{1}$
	\end{itemize}
\end{frame}

\begin{frame}{What happens if we measure two compatible observables $A$ and $B$?}
	\begin{itemize}
		\item If eigenvalues of $A$ are non-degenerate:
		      \[
			      \ket{\alpha} \xrightarrow[A \text{ get } a']{\text{measurement}} \ket{a',b'} \xrightarrow[B \text{ get } b']{\text{measurement}} \ket{a',b'}\xrightarrow[A \text{ get } a']{\text{measurement}}\ket{a',b'}
		      \]
		\item If eigenvalues of $A$ are n-fold degenerate:
		      \[
			      \ket{\alpha} \xrightarrow[A \text{ get } a']{\text{measurement}} \sum_{i=1}^n C_{a}^{i}\ket{a',b^{(i)}} \xrightarrow[B \text{ get } b^{(j)}]{\text{measurement}} \ket{a',b^{(j)}}\xrightarrow[A \text{ get } a']{\text{measurement}}\ket{a',b^{(j)}}
		      \]
	\end{itemize}

	$A$ measurements and $B$ measurements do not interfere for compatible observables.

	Note: Incompatible observables do not have a complete set of simultaneous eigenkets.
\end{frame}

\subsubsection{Uncertainty Relation}
\begin{frame}
	\frametitle{Uncertainty Relation}
	Consider a certain physical state $\ket{\;}$\\
	\[\Delta A = A - \ev{A}\]
	mean square deviation:
	\[
		\ev{(\Delta A)^2} = \ev{ (A^2 - 2A \ev{A} + \ev{A}^2 ) } = \ev{A^2} - \ev{A}^2
	\]
	Note: if the state is an eigenket of A: $\ket{\;} = \ket{a'}$
	\[\begin{split}
			\ev{A} = \mel{a'}{A}{a'} = a' \Rightarrow \ev{(\Delta A)^2} = 0 \\ \therefore \text{mean standard deviation vanishes}
		\end{split}
	\]
\end{frame}
\begin{frame}
	\begin{theorem}[Uncertainty Relation]
		Let $A,B$ be observables. Then, for any state, we have
		\[
			\ev{(\Delta A)^2}\ev{(\Delta B)^2} \ge |\ev{\acomm{A}{B}}|^2
		\]
	\end{theorem}
	Proof

	Lemma 1: Schwartz Inequality:

	\begin{gather*}
		\ip{\alpha} \ip{\beta} \ge |\ip{\alpha}{\beta}|^2 \; \text{i.f.} \; |\va a|^2 |\va b|^2 \ge  |\va a \cdot \va b|^2\\
		a^2 b^2 \ge (ab \cos{\theta})^2\\
		|(\ket \alpha + \lambda \ket \beta)|^2 \ge 0  \quad (\bra{\alpha} + \lambda^* \bra \beta) \cdot (\ket \alpha + \lambda \ket \beta) \ge 0
	\end{gather*}
	where $\lambda \in \mathbb{C}$. Take $\lambda = -\frac{\ip{\beta}{\alpha}}{\ip{\beta}{\beta}} \quad \Rightarrow$ Schwartz Inequality.
\end{frame}
\begin{frame}
	Lemma 2: The expectation value of a Hermitian Operator is real.

	\[
		\mel{a''}{X}{a'} = \mel{a'}{X^\dag}{a''}^*
	\]

	For $\ket{a'} = \ket{a''}$, $\mel{a'}{X}{a'} = \mel{a'}{X}{a'}^*$. Therefore purely real.

	\vfill

	Lemma 3: The expectation value of an anti-Hermitian operator, define by $C=-C^\dag$ is purely imaginary.

	\begin{gather*}
		\mel{a'}{X}{a'} = \mel{a'}{\underbrace{X^\dag}_{-X}}{a'}^*\\
		-\mel{a'}{X}{a'}^*. \qed
	\end{gather*}
\end{frame}

\begin{frame}
	In the Schwartz Inequality:  $ \ket{\alpha} = \Delta A \ket{} \quad \ket\beta = \Delta B \ket{}$

	Since $A,B$ are observables they are Hermitian. $$\ev{(\Delta A)^2}\ev{(\Delta B)^2} \ge |\ev{(\Delta A \Delta B)}|^2$$

	Decomposing ,
	\begin{gather*}
		\Delta A \Delta B = \overbrace{\frac{1}{2}\comm{\Delta A}{\Delta B}}^{\text{anti-Hermitian}} + \overbrace{\frac{1}{2} \acomm{\Delta A}{\Delta B}}^{\text{Hermitian}}
	\end{gather*}
\end{frame}
\begin{frame}
	with $\comm{\Delta A}{\Delta B} = \comm{A }{B }$ being anti-Hermitian.
	\begin{align*}
		(\comm{A}{B})^\dag & = (AB - BA)^\dag = B^\dag A^\dag - A^\dag B^\dag \\
		                   & = BA - AB = -\comm{A}{B}
	\end{align*}
	$\acomm{\Delta A}{\Delta B}$ being Hermitian $\to (\comm{\Delta A}{\Delta B})^\dag = (\Delta A \Delta B + \Delta B \Delta A)^\dag$
	$$ \Delta B \Delta A + \Delta A \Delta B = \acomm{\Delta A}{\Delta B}$$
	\[
		c = ib + a \to |c|^2 = b^2+a^2
	\]
	\[
		|\ev{\Delta A \Delta B}|^2 = \frac14 |\comm{A}{B}|^2 + \underbrace{\frac14 |\ev{\acomm{\Delta A}{\Delta B}}|^2}_{\ge 0 } \ge |\ev{\comm{A}{B}}|^2
	\]

\end{frame}

\subsubsection{Change of Basis}
\begin{frame}
	\frametitle{Change of Basis}
	Two incompatible observables $A$ and $B$. \[A : \{\ket{a'}\} \quad B: \{\ket{b'}\}\]

	A representation: representation in which the base eigenkets are given by $\{ \ket{a'}\}$.

	How to relate, $A$'s  representation to $B$'s representation $\ket{a'} \to \ket{b'}$?

	\begin{theorem}[3]
		Given two sets of base kets with satisfying orthonormality and completeness, there exists a unitary operator
		$U$ such that

		\[
			\ket{b^{(1)}} = U \ket{a^{(1)}},\ket{b^{(2)}} = U \ket{a^{(2)}}, \dots \ket{b^{(N)}} = U \ket{a^{(N)}}
		\]
	\end{theorem}
	An operator is unitary when it satisfies $UU^\dag = U^\dag U = \mathbb{1}$

	Proof :
	\[
		U= \sum_k \op{b^{(k)}}{a^{(k)}} \qed
	\]

\end{frame}

\begin{frame}{Matrix representation of $U$ in the old $\{\ket{a'}\}$ basis}
	\[
		\mel{a^{(k)}}{U}{a^{(l)}} = \ip{a^{(k)}}{b^{(k)}}
	\]
	Given an arbitrary ket $\ket{\alpha}$.
	\begin{gather*}
		\ket{\alpha} = \sum_{a'} \ket{a'} \bra{a'}\ket{\alpha}\\
		\intertext{where $\ip{a'}{\alpha}$ is the coefficient for representing $\ket{\alpha}$ in $\ket{a'}$ basis.}
		\ip{b^{(k)}}{\alpha} = \sum_l \ip{b^{(k)}}{a^{(l)}} \ip{a^{(l)}}{\alpha} = \sum_l \ip{a^{(l)}}{b^{(k)}}^* \ip{a^{(l)}}{\alpha}\\
		\ket{b^{(k)}} = U \ket{a^{(k)}} \leftrightarrow \bra{b^{(k)}} = \bra{a^{(k)}}U^\dag\\
		\ip{b^{(k)}}{\alpha} = \sum_l \mel{a^{(k)}}{U^\dag}{a^{(l)}}\ip{a^{(l)}}{\alpha}\\
		\mqty(\text{Column vector of }\\ \ip{b^{(k)}}{\alpha}) = \mqty(\text{matrix of } \\ U^\dag \\ \text{ in } A \text{ representation}) \mqty(\text{Column vector of }\\ \ip{a^{(l)}}{\alpha})
	\end{gather*}
\end{frame}

\begin{frame}
	Relate the old matrix elements of an opertator $X$ to the new matrix elements of $X$:

	\begin{gather*}
		\mel{b^{(k)}}{X}{b^{(\ell)}} = \mel{b^{(k)}}{\overbrace{\mathbb{1}}^{\sum_{a'} \op{a'}} \cdot X \cdot \overbrace{\mathbb{1}}^{\sum_{a'} \op{a'}}}{b^{(\ell)}}\\
		= \sum_m \sum_n \ip{b^{(k)}}{a^{(m)}}\mel{a^{(m)}}{X}{a^{(n)}} \ip{a^{(n)}}{b^{(\ell)}}\\
		\underbrace{\mel{b^{(k)}}{X}{b^{(\ell)}}}_{\text{matrix in $b$ rep.}} = \sum_m \sum_n \underbrace{\mel{a^{(k)}}{U^\dag}{a^{(m)}} \mel{a^{(m)}}{X}{a^{(n)}} \mel{a^{(n)}}{U}{a^{(\ell)}}}_{\text{matrices represented in the $a$ basis}}
	\end{gather*}

	$$ X' = U^\dag X U \qquad{\to} \quad \text{similarity transformation}$$
\end{frame}

\begin{frame}{Trace}
	\[\Tr(X) = \sum_{a'} \mel{a'}{X }{a'}\] trace is independent of representation.

	\begin{align*}
		\sum_{a'} \mel{a'}{X }{a'} & = \sum_{b' b'' a'} \ip{a'}{b'} \mel{b'}{X}{b''} \ip{b''}{a'} \\
		                           & = \sum_{b' b'' a'} \ip{b''}{a'} \ip{a'}{b'} \mel{b'}{X}{b''} \\
		                           & = \sum_{b' b''} \ip{b''}{b'} \mel{b'}{X}{b''}                \\
		                           & = \sum_{b'} \ev{X}{b'}
	\end{align*}
\end{frame}

\begin{frame}{Properties of the Trace}
	\begin{itemize}
		\item $\Tr(XY) = \Tr(YX)$
		\item $\Tr(XYZ) = \Tr(ZXY) = \Tr(YZX)$
		\item $\Tr(U^\dag X U) = \Tr(X)$
		\item $\Tr(\op{b'}{a'}) = \ip{a'}{b'}$
	\end{itemize}
\end{frame}

\begin{frame}
	If we know the $A$ representation of an operator $B$, how do we find the eigenvalues and eigenkets of $B$?
	\begin{gather*}
		B \ket{b'} = b'\ket{b'} \intertext{Now we put a $\bra{a''}$ and we use completeness.}
		\sum_{a'} \mel{a''}{B }{a'}\ip{a'}{b'} = b'\ip{a''}{b'}
		\intertext{Take $B$ to be the $\ell$'th eigenket of B}
		\mqty[\xmat*{B}{2}{3} & \dots \\ \vdots & \vdots & \ddots] \mqty[C_1^{(\ell)} \\ C_2^{(\ell)} \\ \vdots] = b^{(\ell)} \mqty[C_1^{(\ell)}\\ C_2^{(\ell)} \\ \vdots]
	\end{gather*}
	Where $B_{ij} = \mel{a^{(i)}}{B}{a^{(j)}} \qq{;} C_k^{(\ell)} = \ip{a^{(k)}}{b^{(\ell)}}$
\end{frame}

\begin{frame}
	Non-trivial solution for $C_k^{(\ell)}$ requires $\det(B - \lambda \mathbb{1}) = 0 \Rightarrow N $ values of $\lambda$: \\
	$$b^{(1)},b^{(2)},\dots,b^{(\ell)} \qq{--- eigenvalues}$$ \\
	\[
		\mqty[C_1^{(1)}\\ C_2^{(1)} \\ \vdots], \mqty[C_1^{(2)}\\ C_2^{(2)} \\ \vdots], \dots, \mqty[C_1^{(N)}\\ C_2^{(N)} \\ \vdots] \qq{--- eigenkets}
	\]
	since $C_k^{(\ell)} = \mel{a^{(k)}}{U}{a^{(\ell)}}$, then having solved for the eigenvectors means having solved for U!

	\[
		U^\dag A U = \mqty(\dmat{\ddots,\ddots,\ddots})
	\]
\end{frame}


\begin{frame}
	Finally for basis states $\{\ket{a'}\} \& \{\ket{b'}\}$ where $\ket{b'} = U \ket{a'}$

	\begin{gather*}
		A\ket{a^{\ell}} = a^{(\ell)} \ket{a^{(\ell)}}\\
		U A \ket{a^{(\ell)}} = a^{(\ell)} U \ket{a^{(\ell)}}
		\intertext{Now inserting $U^{-1} U = \mathbb{1}$ after UA on the LHS (note: $U^{-1} = U^\dag$)}
		(UAU^{-1}) \underbrace{U\ket{a^{(\ell)}}}_{\ket{b^{(\ell)}}} = a^{(\ell)} \underbrace{U\ket{a^{(\ell)}}}_{b^{(\ell)}}\\
		(UAU^{-1}) \ket{b^{(\ell)}} = a^{(\ell)} \ket{b^{(\ell)}}
	\end{gather*}

	Therefore $A$ and $UAU^{-1}$ share the same eigenvalue spectra.
\end{frame}


\subsection[$x,p$ and Translation]{Position, Momentum and Translation}
\begin{frame}{Position, Momentum and Translation}{Continuous Spectra}
	\small
	\begin{itemize}
		\item Discrete eigenvalue, eg. $S_z$, dimensionality of the state space can be finite or infinite.
		\item Continuous eigenvalue spectra, eg. $p_z$, dimensionality of the state space is infinite.
	\end{itemize}
	% \tiny
	\begin{align*}
		\text{Discrete spectra}                                       &  & \text{Continuous Spectra}                                              \\
		A\ket{a'} = a'\ket{a'}                                        &  & \xi \ket{\xi'} = \xi'\ket{\xi'}                                        \\
		\ip{a'}{a''} = \delta_{a'a''}                                 &  & \ip{\xi'}{\xi''} = \delta(\xi'- \xi'')                                 \\
		\sum_{a'} \op{a'} = \mathbb{1}                                &  & \int \dd{\xi'} \op{\xi'} = \mathbb 1
		\intertext{For $\ip{\alpha}  = 1$}
		\sum_{a'} \qty|\ip{a'}{\alpha}|^2 = \mathbb 1                 &  & \int \dd{\xi'} \qty|\ip{\xi'}{\alpha}|^2 = \mathbb 1                   \\
		\ip{\beta}{\alpha} = \sum_{a'} \ip{\beta}{a'} \ip{a'}{\alpha} &  & \ip{\beta}{\alpha} = \int \dd{\xi'} \ip{\beta}{\xi'} \ip{\xi'}{\alpha} \\
		\mel{a''}{A}{a'} = a' \delta_{a' a''}                         &  & \mel{\xi''}{\xi}{\xi'} = \xi' \delta(\xi'' - \xi')
	\end{align*}


\end{frame}

\subsubsection[Position Eigenkets]{Position eigenkets and position measurements}
\begin{frame}{Position eigenkets and position measurements}
	How to extend the idea of measurements to measurements of observables with continuous spectra?
	Consider the measurement of a particle's position in 1-D:
	\[
		x\ket{x'}=x'\ket{x'}
	\]
	Before measurement
	$$ \ket{\alpha} = \int_{-\infty}^\infty \dd{x'} \op{x'}\ket{\alpha}$$
\end{frame}

\begin{frame}
	Detector: will click only when  the particle is precisely at $x'$: $$ \ket{\alpha} \to \ket{x'}$$
	Discrete spectra case:
	\[
		\ket{\alpha} = \sum_{a'} \ket{a'}\ip{a'}{\alpha}
	\]
	The probability of obtaining $a'$ from a measurement is $\qty|\ip{a'}{\alpha}|^2$.

	Continuous spectra case:

	\[
		\ket{\alpha} = \int_{-\infty}^\infty \dd{x'} \ket{x'}\ip{x'}{\alpha}
	\]
	The probability of obtaining $x'$ from measurement is $\qty|\ip{x'}{\alpha}|^2 \dd{x'}$ and $\qty|\ip{x'}{\alpha}|^2$ is the probability density (probability per unit length).

	\begin{gather*}
		\ip{\alpha} = 1 = \int_{-\infty}^\infty \dd{x'} \bra{\alpha}\ket{x'}\ip{x'}{\alpha}\\
		= \int_{-\infty}^\infty \dd{x'} \qty|\ip{x'}{\alpha}|^2 = 1
	\end{gather*}
\end{frame}

\begin{frame}{Generalize to 3-D}
	\begin{gather*}
		\ket{\alpha} = \int_{-\infty}^\infty \dd[3]{x'} \ket{x'}\ip{x'}{\alpha}
		\intertext{where $\ket{x'}$ is a simultaneous eigenket of $X,Y,Z$}
		\ket{\vb x'} = \ket{x',y',z'}
		\intertext{with}
		X\ket{\vb x'}= x' \ket{\vb x'}\\
		Y\ket{\vb x'} = y' \ket{\vb x'}\\
		Z \ket{\vb x'} = z' \ket{\vb x'}
	\end{gather*}
	assumes $x,y,z$ are compatible observables i.e. $\comm{x_i}{x_j}= 0$. Where $i,j = \{1,2,3\} $ standing for $\{x,y,z\}$ respectively.
\end{frame}

\subsubsection{Translation}
\begin{frame}
	\frametitle{Translation}

	Change a state localized around $\vb{x'}$ to another state around $\vb{x'} + \dd{\vb{x'}}$
	\\
	\begin{align*}
		\mathcal{J}(\dd{\vb x'}) \ket{\vb x'} & = \ket{\vb x' + \dd{\vb x'}}                                       \\
		\ket{\alpha}                          & \to \mathcal{J}(\dd{\vb x'})\ket{\alpha}                           \\
		                                      & = \mathcal{J}(\dd{\vb x'}) \int \dd[3]{x'} \op{\vb x'}\ket{\alpha} \\
		                                      & = \int \dd[3]{x'} \op{\vb x' + \dd{\vb x'}}{\vb x'} \ket \alpha    \\
		                                      & = \int \dd[3]{x'} \op{\vb x'}{\vb x' - \dd{\vb x'}} \ket \alpha
	\end{align*}


\end{frame}

\begin{frame}
	\frametitle{Translation}
	What properties must be satisfied by $\mathcal{J}(\dd{\vb x'})$?
	\begin{itemize}
		\item $\ket{\alpha}$ is normalized $\to \mathcal{J}(\dd{\vb x'}) \ket \alpha$ should be also normalized.
		      \begin{gather*}
			      \ip{\alpha} = 1 = \ev{\mathcal{J}^\dag (\dd{\vb x'}) \mathcal{J}(\dd{\vb x'})}{\alpha}\\
			      \Rightarrow \mathcal{J}^\dag \mathcal J= \mathbb{1}
		      \end{gather*}
		      i.e $\mathcal{J}$ is a unitary operator.

		\item $\mathcal{J}(\dd{\vb x'})\mathcal{J}(\dd{\vb x''}) = \mathcal{J}(\dd{\vb x'} + \dd{\vb x''})$
		\item $ \mathcal{J}(-\dd{\vb x'}) = \mathcal{J}^{-1}(\dd{\vb x'})$
		\item $ \displaystyle \lim_{\dd{\vb x'} \to 0} \mathcal{J}(\dd{\vb x'}) = 1$ a.k.a expect the difference between $\mathcal{J}(\dd{\vb x'})$ and $\mathbb 1$ to be first order in $\dd{x'}$ i.e. $\mathcal{J}(\dd{\vb x'}) = \mathbb{1} + \Delta(\dd{\vb x'})$.
	\end{itemize}

\end{frame}

\begin{frame}{Translation}
	All of the above properties are satisfied by taking
	\[
		\mathcal{J}(\dd{\vb x'}) = \mathbb 1 - i\vb{K} \cdot \dd{\vb x'}
	\]
	where the components of $\vb K$ ($K_x, K_y, K_z$) are Hermitian operators.
	Checking the unitarity property
	\begin{align*}
		\mathcal{J}^\dag(\dd{\vb x'}) \mathcal{J}(\dd{\vb x'}) & = (\mathbb 1 + i\vb K \cdot  \dd{\vb x'})(\mathbb 1 - i\vb K \cdot  \dd{\vb x'}) \\
		                                                       & = \mathbb{1} - i(\vb K - \vb K)\cdot \dd{\vb x'}                                 \\
		                                                       & = \mathbb{1} + \order{[(\dd{\vb x'})^2]} \simeq \mathbb 1
	\end{align*}

	Consider
	\begin{gather*}
		\vb x \mathcal{J}(\dd{\vb x'})\ket{\vb x'} = \vb x	\ket{\vb x' + \dd{\vb x'}} = (\vb x' + \dd{\vb x'})\ket{\vb x' + \dd{\vb x'}}\\
		\mathcal{J}(\dd{\vb x'}) \vb x \ket{\vb x'} = \vb x' \mathcal{J}(\dd{\vb x'}) \ket{\vb x'} = \vb x' \ket{\vb x' + \dd{\vb x'}}\\
		\Rightarrow \comm{\vb x}{\mathcal{J}(\dd{\vb x'})} \ket{\vb x'} = \dd{\vb x'} \ket{\vb x' + \dd{\vb x'}} \simeq \dd{\vb x'} \ket{\vb x'}\\
		\comm{\vb x}{\mathcal{J}(\dd{\vb x'})}  = \dd{\vb x'}
	\end{gather*}

\end{frame}

\begin{frame}
	\frametitle{Translation (cont)}
	\begin{gather*}
		\comm{\vb x'}{\mathcal{J}} = \dd{\vb x'}\\
		-i\vb x' (\vb K \cdot \dd{\vb x'}) + i(\vb K \cdot \dd{\vb x'})\vb x' = \dd{\vb x'}
	\end{gather*}
	By choosing $\dd \vb x'$ in the direction of $\vu x_j$ and forming the scalar product with $\vu x_i$, we obtain
	\[
		-ix_i k_j \dd x' + ik_j \dd x' x_i  = \dd x' \vu x_i \vu x_j =\dd x' \delta_{ij} \to \comm{x_i}{k_j} = i \delta_{ij}
	\]

\end{frame}

\begin{frame}
	\frametitle{Momentum as a generator of translation}
	$\vb K$ is the wave number operator. $\vb K = \frac{\vb p}{\hbar}$, where $\vb p$ is the momentum operator.

	\begin{gather*}
		\mathcal{J}(\dd \vb x') = \mathbb{1} - i \frac{\vb p}{\hbar} \cdot \dd{\vb x'}\\
		\comm{x_i}{p_j} = i \hbar \delta_{ij}
	\end{gather*}

	eg. $x,p_x$ are incompatible observables, $x,p_y$ are compatible observables.
	It is therefore impossible to find simultaneous eigenkets of $x$ and $p_x$, and thus we can obtain the position
	momentum uncertainty relation of W. Heisenberg.

	$$
		\ev{(\Delta x)^2} \ev{(\Delta p_x)^2} \ge \frac{\hbar^2}{4}
	$$
\end{frame}


\begin{frame}
	\frametitle{Momentum as a generator of translation (cont)}
	Consider a finite translation in the $x$-direction by $\Delta x'$.
	\[ \mathcal{J}(\Delta x' \vu x)	\ket{\vb x'} = \ket{\vb x' + \Delta x' \vu x} \]
	with
	$$
		\mathcal{J}(\Delta x' \vu x) = \lim_{N \to \infty} \qty(\mathbb{1} - \frac{i p_x \Delta x'}{N \hbar})^N
		= \exp(-\frac{i p_x \Delta x'}{\hbar})
	$$
	For any operator $x$,
	\[\exp(x) = \mathbb{1} + x + \frac{x^2}{2} + \dots = \sum_{N}^{\infty} \frac{x^n}{n!}\]

\end{frame}


\frame{LOTS MORE TO ADD!}
