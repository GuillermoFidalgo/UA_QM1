\section{Base Kets and Matrix Representations}
\begin{frame}{Eigenkets of an observation}
	\begin{alertblock}{Theorem 1}
		The eigenvalues of a Hermitian operator $A$ are real; the eigenkets of $A$ corresponding to different eigenvalues are orthogonal.
	\end{alertblock}
	Proof:

	\begin{gather}
		A\ket{a'} = a'\ket{a'} \label{eq:1}\\
		\bra{a'' }A = a''^* \bra{a''} \label{eq:2}\\
		\bra{a''} \cdot \ref{eq:1} - \ref{eq:2}\cdot \ket{a'} \to (a'-a'') \ip{a''}{a'} = 0
	\end{gather}
	If $a'=a''$, then $a'=a''^*=a'^*$ since $\ip{a'}{a'} = 0$

	$\therefore a'$ is real. If $a'\ne a'', a'-a'' \ne 0 \Longrightarrow \ip{a''}{a'} = 0$

	i.e. $\ket{a''} \& \ket{a'}$ are orthogonal.

\end{frame}
\begin{frame}
	Can normalize $\ket{a'}$ so that $\{\ket{a'}\}$ forms an orthogonal set with $\ip{a''}{a'}= \delta_{a' a''}$.

	The set of eigenkets forms a complete set to span the ket space.
\end{frame}

\subsection{Eigenkets and Base Kets}
\begin{frame}
	\frametitle{Eigenkets and Base Kets}

	\begin{equation}\label{eq:alpha base}
		\ket{\alpha} = \sum_{a'} C_{a'} \ket{a'}
	\end{equation}
	--- expansion of an arbitrary ket in the ket space spanned by the eigenkets of $A$.
	\begin{gather*}
		\bra{a''} (\ref{eq:alpha base}): \quad \ip{a''}{\alpha} = \sum_{a'} C_{a'} \underbrace{\ip{a''}{a'}}_{\delta_{a'' a'}} = C_{a''}\\
		\Rightarrow C_{a'} = \ip{a'}{\alpha}\\
		\therefore \quad \ket{\alpha} = \sum_{a'} \op{a'}\ket{\alpha}\\
		\text{C.f.} \quad \vec{V} = \sum_i \hat e_i (\hat e_i \cdot \vec V) \qq{for a vector} \vec V
	\end{gather*}
\end{frame}


\begin{frame}
	\frametitle{Completeness relation or closure relation}
	$$
		\mathbb{1} = \sum_{a'} \op{a'}
	$$
	eg.
	\begin{columns}
		\column{.5\textwidth}
		\begin{gather*}
			\ip{\alpha} = \bra{\alpha} \cdot \qty(\sum_{a'}\op{a'}) \cdot \ket{\alpha}	\\
			= \sum_{a'} \ip{\alpha}{a'}\ip{a'}{\alpha}\\
			= \sum_{a'} \ip{a'}{\alpha}^*\ip{a'}{\alpha}\\
			= \sum_{a'} \qty|\ip{a'}{\alpha}|^2
		\end{gather*}
		\column{.5\textwidth}
		also if $\ket{\alpha}$ is normalized then $\ip{\alpha}  = 1$
		\begin{align*}
			\Rightarrow & \sum_{a'} \qty|\ip{a'}{\alpha}|^2 \\
			            & = \sum_{a'} |C_{a'}|^2            \\
			            & = 1
		\end{align*}
	\end{columns}
\end{frame}
\begin{frame}
	\frametitle{Projection operator}

	Note: $\op{a'}$ is a projection operator because
	\[
		\qty(\op{a'})\cdot \ket{\alpha} = \ket{a'} \ip{a'}{\alpha} = C_{a'} \ket{a'}
	\]
	c.f. Dyadic ooperator in 3D Euclidean space
	\[\hat e_i \hat e_j \qquad (\hat x \hat x) \cdot \bar V = \hat x (\hat x \cdot \bar V)\]
\end{frame}
\subsection{Matrix Representations}
\begin{frame}
	\frametitle{Matrix Representations}
	A ket can be represented as a column vector.\\
	An operator can be represented as a matrix.
	\[
		\ket \alpha = \sum_{a'} \ket{a'} \ip{a'}{\alpha}
	\]
	where $a'$ is a base ket.

	If we use a column vector to represent $\ket{a'}$ it will be the following. Denote
	\begin{gather*}
		\qty{a'} = \qty{a^{(1)},a^{(2)},\dots}\\
		\ket{a^{(1)}} = \mqty[1 \\ 0 \\ 0 \\ \vdots] \quad \ket{a^{(2)}} = \mqty[0 \\ 1 \\ 0 \\ \vdots]
	\end{gather*}

\end{frame}
\begin{frame}
	\frametitle{Matrix Representations}

	So
	\[
		\ket{\alpha} = \ket{a^{(1)}} = \mqty[\ip{a^{(1)}}{\alpha} \\ \ip{a^{(2)}}{\alpha} \\ \vdots] = \mqty[C_{a^{(1)}}\\C_{a^{(2)}}\\ \vdots]
	\]
	Operator $X$ in matrix form
	\[
		X = \mathbb{1} \cdot X \cdot \mathbb{1} = \sum_{a'} \op{a''} X\sum_{a'} \op{a'}  = \sum_{a'' a'} \op{a''} X \op{a'}
	\]
	\[
		X = \mqty[
			\ev{X}{a^{(1)}} & \mel{a^{(1)}}{X}{a^{(2)}} & \cdots \\
			\mel{a^{(2)}}{X}{a^{(1)}} & \ev{X}{a^{(2)}} & \cdots \\
			\vdots & \vdots & \ddots
		]
	\]
	is a square matrix
	Note:

	\[
		\mel{a'' }{X}{a'} = \mel{a'}{X^\dag}{a''}^* \Rightarrow \mel{a'}{X^\dag}{a''} = \mel{a''}{X}{a'}^*
	\]
	Hermitian adjoint corresponds to taking the transpose and complex conjugate of the matrix form of the operator.
\end{frame}

\begin{frame}
	\frametitle{Matrix Representations}

	\begin{gather*}
		Z = XY\\
		\mel{a''}{Z}{a'} = \mel{a''}{XY}{a'} \\
		= \sum_{a'''} \mel{a''}{X}{a'''} \mel{a'''}{Y}{a'}\\
		\ket{\gamma} = X \ket{\alpha}\\
		\ip{a'}{\gamma} = \mel{a'}{X}{\alpha} \Rightarrow \sum_{a''} \mel{a'}{X}{a''}\ip{a''}{\alpha} \\
		\mqty[\ip{a^{(1)}}{\gamma} \\ \ip{a^{(2)}}{\gamma} \\ \vdots] = \mqty[
			\ev{X}{a^{(1)}} & \mel{a^{(1)}}{X}{a^{(2)}} & \cdots \\
			\mel{a^{(2)}}{X}{a^{(1)}} & \ev{X}{a^{(2)}} & \cdots \\
			\vdots & \vdots & \ddots
		] \mqty[\ip{a^{(1)}}{\alpha} \\ \ip{a^{(2)}}{\alpha} \\ \vdots]
	\end{gather*}
	same for bras
\end{frame}

\begin{frame}{Matrix Representations}
	\begin{gather*}
		\bra{\gamma} = \bra{\alpha} X \\
		\ip{\gamma}{a'} = \mel{\alpha}{X}{a'} = \sum_{a''} \ip{\alpha}{a''}\mel{a'' }{X}{a'}\\
		= \mqty[\ip{\gamma}{a^{(1)}} & \ip{\gamma}{a^{(2)}} & \cdots]  \\
		= \mqty[\ip{\alpha}{a^{(1)}} & \ip{\alpha}{a^{(2)}} & \cdots]
		\mqty[
			\ev{X}{a^{(1)}} & \mel{a^{(1)}}{X}{a^{(2)}} & \cdots \\
			\mel{a^{(2)}}{X}{a^{(1)}} & \ev{X}{a^{(2)}} & \cdots \\
			\vdots & \vdots & \ddots
		]
	\end{gather*}

\end{frame}

\begin{frame}{Matrix Representations}
	For inner products
	\begin{gather*}
		\ip{\beta}{\alpha} = \sum_{a'} \ip{\beta}{a'}\ip{a'}{\alpha} \\
		= \mqty[ \ip{\beta}{a^{(1)}} & \ip{\beta}{a^{(2)}} & \cdots]
		\mqty[ \ip{a^{(1)}}{\alpha} \\
			\ip{a^{(2)}}{\alpha} \\
			\vdots]
		\\
	\end{gather*}
	For outer products
	\begin{gather*}
		\op{\beta}{\alpha} = \qty(\sum_{a''} \op{a''}) \ket\beta \bra{\alpha} \qty(\sum_{a'} \op{a'})=\sum_{a' a''} \op{a''}\op{\beta}{\alpha} \op{a'} \\
		= \ip{a'''}{\beta} \ip{\alpha}{a'''} = \mqty[
			\ip{a^{(1)}}{\beta}\ip{a^{(1)}}{\alpha}^* & \ip{a^{(1)}}{\beta}\ip{a^{(2)}}{\alpha}^* & \cdots \\
			\ip{a^{(2)}}{\beta}\ip{a^{(1)}}{\alpha}^* & \ip{a^{(2)}}{\beta}\ip{a^{(1)}}{\alpha}^* & \cdots \\
			\vdots & \vdots & \ddots
		]
	\end{gather*}
\end{frame}

\begin{frame}{Matrix Representations}
	\begin{itemize}
		\item If $\{\ket{a'}\}$ is the set of eigenkets of the operator $A = \mathbb{1}\cdot A \cdot \mathbb{1}$
	\end{itemize}
	\begin{gather*}
		\sum_{a'' a'} \op{a''} \underbrace{A\ket{a'}}_{a'\ket{a'}}\bra{a'} = \sum_{a' a''} a'\ket{a'}\underbrace{\ip{a''}{a'}}_{\delta_{a'' a'}} \ket{a'} = \sum_{a'} a'\op{a'}\\
		\mqty[\dmat{ a^{(1)},a^{(2)},\ddots }]
	\end{gather*}
\end{frame}

\begin{frame}{Example: Spin 1/2 systems}
	Base kets : $ \ket{S_z;\pm} = \ket{\pm}$\\
	Completeness: $\mathbb{1} = \op{+} + \op{-}$
	\begin{gather*}
		\text{Eigenvalue eqn}: S_z \ket{\pm} = \pm \frac\hbar2 \ket{\pm}\\
		S_z = \frac\hbar2 \op{+} - \frac{\hbar}{2}\op{-} = \frac{\hbar}{2} \qty(\op{+} - \op{-})
	\end{gather*}
\end{frame}

\subsection{Example: Spin 1/2 systems}
\begin{frame}{Example: Spin 1/2 systems}

	Now define

	\[
		\begin{split}
			S_+ = \hbar \op{+}{-} \\
			S_- = \hbar \op{-}{+}
		\end{split}\Biggr\} \text{Ladder Operators}
	\]

	Note:
	\[
		\begin{split}
			S_+ \ket{-} = \hbar \ket + \overbrace{\ip{-}}^1 = \hbar\ket + \\
			S_+ \ket{+} = \hbar \ket + \underbrace{\ip{-}{+}}_0 =  0      \\
		\end{split} \Biggr\} \text{Raising Operator}
	\]
	%
	\[
		\begin{split}
			S_- \ket{+} = \hbar \ket - \overbrace{\ip{+}}^1 = \hbar\ket - \\
			S_- \ket{-} = \hbar \ket - \underbrace{\ip{+}{-}}_0 =  0      \\
		\end{split} \Biggr\} \text{Lowering Operator}
	\]

\end{frame}


\begin{frame}{Matrix Representations}{Spin 1/2 systems}
	$$\ket{+} = \mqty[1 \\ 0] \qquad \ket- = \mqty[0 \\ 1]$$
	\begin{gather*}
		S_z \doteq \frac{\hbar}{2} \pqty{\mqty[1\\0] \mqty[1 & 0] - \mqty[0 \\ 1] \mqty[0 & 1]} = \frac{\hbar}{2} \mqty[\pmat{3}]\\
		S_+ = \hbar \op{+}{-} = \hbar \mqty[1 \\ 0 ] \mqty[0 & 1] = \hbar \mqty[0 & 1\\0 & 0]\\
		S_- = \hbar \mqty[0&0\\1&0]
	\end{gather*}
\end{frame}


\section{Measurements, Observables and the Uncertainty Relations}
\begin{frame}{Measurements, Observables and the Uncertainty Relations}{Measurements}
	\begin{itemize}
		\item Observable A \\
		      Before measurement, the system is in a state $\ket \alpha$
		      \[
			      \ket{\alpha} = \sum_{a'} \ket{a'} \ip{a'}{\alpha}
		      \]
		      where $\{\ket{a'}\}$ is the set of eigenkets for $A$. Under measuerment the system is thrown into an eigenstate
		      \[ \ket{\alpha} \xrightarrow{\text{ \tiny A measurement}} \ket{a'}\]
		      If the state before measurement is an eigenstate of the observables, say $\ket{a'}$
		      \[ \ket{a'} \xrightarrow{\text{\tiny A measurement}} \ket{a'}\]
	\end{itemize}
\end{frame}

\begin{frame}
	\begin{itemize}
		\item Probability for jumping into some particular state $\ket{a'}$ is $|\ip{a'}{\alpha}|^2$ where $\ket{\alpha}$ is normalized.
		      In the particular case when $\ket{\alpha} = \ket{a'}$ before measurement, then probability of measuring $A$ to be $a'$ will be 1 since
		      \[\qty|\ip{a'}|^2 = 1\]
		      Probability of measuring $A$ to be $a''$ (where $a'' \neq a'$) will be 0 since $\qty|\ip{a''}{a'}|^2 = 0$ due to orthogonality.
		\item Expectation value of $A$ when the system is in state $\ket \alpha$. \[\ev{A} =\ev{A}{\alpha} \]
		      means the average measured value
		      \begin{gather*}
			      \ev{A} = \sum_{a''}\sum_{a'} \ip{\alpha}{a''}\mel{a''}{A}{a'}\ip{a'}{\alpha}\\
			      = \sum_{a''}\sum_{a'} a'\ip{\alpha}{a''} \underbrace{\ip{a''}{a'}}_{\delta_{a''a'}} \ip{a'}{\alpha}\\
			      = \sum_{a'} a' \ip{\alpha}{a'}\ip{a'}{\alpha} = \sum_{a'} a' \ip{a'}{\alpha} \ip{a'}{\alpha}^* = \sum_{a'} a'|\ip{a'}{\alpha}|^2
		      \end{gather*}
	\end{itemize}
\end{frame}

\subsection{Compatible Observables}
\begin{frame}{Compatible Observables}
	\begin{itemize}
		\item Compatible observables if $\comm{A}{B} = 0$, otherwise they are called incompatible. e.g.
		      \begin{gather*}
			      [S^2,S_z] = 0 \Rightarrow S^2,S_z \; \text{are compatible}\\
			      \comm{S_x}{S_z} \neq 0 \Rightarrow S_x,S_z \; \text{are incompatible}
		      \end{gather*}
		\item Suppose there are two (or more) linearly independent eigenkets of $A$ having the same eigenvalue, then the eigenvalues of the two eigenkets are said to be degenerate. e.g. $\left\{ a^{(1)}, a^{(2)},\dots,a^{(k)}\right\}$ correspond to $\left\{\ket{a^{(1)}},\ket{a^{(2)}},\dots,\ket{a^{(k)}}	\right\}$
		      If $a^{(1)} = a^{(2)} = a, \ket{a^{(1)}} \& \ket{a^{(2)}}$ are degenerate states $\Rightarrow \{a,a,a^{(3)} ,\dots\}$. How to distinguish them?
	\end{itemize}
\end{frame}

\begin{frame}{Compatible Observables}
	\begin{alertblock}{Theorem 2}
		Suppose $A \ \&\  B$ are compatible observables, and the eigenvalues of $A$ are non-degenerate. Then, the matrix elements $\mel{a''}{B}{a'}$ are all diagonal.
	\end{alertblock}

	Proof:
	\begin{align*}
		\mel{a''}{\comm{A}{B}}{a'} & = \mel{a''}{(AB - BA)}{a'}    \\
		                           & = \mel{a''}{(a''B -B a')}{a'} \\
		                           & = (a'' - a') \mel{a''}{B}{a'} \\
		                           & = 0
	\end{align*}
	$\mel{a''}{B}{a'} = 0 $ unless $a'' = a'$
	\[
		\mqty[\ddots\admat{0,\ddots,0}\ddots]
	\]
\end{frame}

\begin{frame}
	If $\comm{A}{B} = 0$ and eigenvalues of $A$ are non-degenerate $ \mel{a''}{B}{a'} $ are diagonal.
	\begin{gather*}
		\mel{a''}{B}{a'}  = \delta_{a'a''} \ev{B}{a'}\\
		B = \mathbb{1}\cdot B \cdot \mathbb{1} = \sum_{a' a''} \ket{a''} \underbrace{\mel{a''}{B}{a'} }_{\delta_{a'a'} \ev{B}{a'}}\bra{a'}\\
		\sum_{a''}\ket{a''}\mel{a''}{B}{a''} \bra{a''}\\
		B\ket{a'} = \sum_{a''} \ket{a''}\mel{a''}{B}{a''}\underbrace{\ip{a''}{a'}}_{\delta_{a'a''}}\\
		= \ket{a'} \underbrace{\mel{a'}{B}{a'}}_{b'} = b'\ket{a'} \; \to \text{eigenvalue eq. for }B.
	\end{gather*}
	$\ket{a'}$ is a simultaneous eigenket of $A$ and $B$ labeled as $\ket{a',b'}$
	\begin{align*}
		A\ket{a',b'} & =a'\ket{a',b'} \\
		B\ket{a',b'} & =b'\ket{a',b'}
	\end{align*}
\end{frame}

\begin{frame}
	If we have a maximal set of mutually commuting observables $A,B,\dots$ with $\comm{A}{B}=\comm{B}{C}=\comm{A}{C}= 0$.\\
	Simultaneous eigenkets are labeled as $\ket{k'} = \ket{a',b',c',\dots}$.
	\begin{itemize}
		\item Orthonormality: $\ip{k''}{k'} = \delta_{k''k'}=\delta_{a''a'}\delta_{b''b'}\delta_{c''c'}$
		\item Completeness: $\sum_{k'}\op{k'} = \sum_a' \sum_b' \sum_c' \dots \op{a',b',c',\dots} = \mathbb{1}$
	\end{itemize}
\end{frame}

\begin{frame}{What happens if we measure two compatible observables $A$ and $B$?}
	\begin{itemize}
		\item If eigenvalues of $A$ are non-degenerate:
		      \[
			      \ket{\alpha} \xrightarrow[A \text{ get } a']{\text{measurement}} \ket{a',b'} \xrightarrow[B \text{ get } b']{\text{measurement}} \ket{a',b'}\xrightarrow[A \text{ get } a']{\text{measurement}}\ket{a',b'}
		      \]
		\item If eigenvalues of $A$ are n-fold degenerate:
		      \[\tiny \hspace{-4mm}
			      \ket{\alpha} \xrightarrow[A \text{ get } a']{\text{measurement}} \sum_{i=1}^n C_{a}^{i}\ket{a',b^{(i)}} \xrightarrow[B \text{ get } b^{(j)}]{\text{measurement}} \ket{a',b^{(j)}}\xrightarrow[A \text{ get } a']{\text{measurement}}\ket{a',b^{(j)}}
		      \]
	\end{itemize}

	$A$ measurements and $B$ measurements do not interfere for compatible observables.

	Note: Incompatible observables do not have a complete set of simultaneous eigenkets.
\end{frame}

\subsection{Uncertainty Relation}
\begin{frame}
	\frametitle{Uncertainty Relation}
	Consider a certain physical state $\ket{\;}$\\
	\[\Delta A = A - \ev{A}\]
	mena square deviation:
	\[
		\ev{(\Delta A)^2} = \ev{ (A^2 - 2A \ev{A} + \ev{A}^2 ) } = \ev{A^2} - \ev{A}^2
	\]
	Note: if the state is an eigenket of A: $\ket{\;} = \ket{a'}$
	\[\begin{split}
			\ev{A} = \mel{a'}{A}{a'} = a' \Rightarrow \ev{(\Delta A)^2} = 0 \\ \therefore \text{mean standard deviation vanishes}
		\end{split}
	\]
\end{frame}
\begin{frame}
	\begin{alertblock}{Uncertainty Relation}
		let $A,B$ be observables. Then, for any state, we have
		\[
			\ev{(\Delta A)^2}\ev{(\Delta B)^2} \ge |\ev{\acomm{A}{B}}|^2
		\]
	\end{alertblock}
	Proof. Lemma 1: Schwartz Inequality:

	\begin{gather*}
		\ip{\alpha} \ip{\beta} \ge |\ip{\alpha}{\beta}|^2 \; \text{i.f.} \; |\va a|^2 |\va b|^2 \ge  |\va a \cdot \va b|^2\\
		a^2 b^2 \ge (ab \cos{\theta})^2\\
		|(\ket \alpha + \lambda \ket \beta)|^2 \ge 0  \quad (\bra{\alpha} + \lambda^* \bra \beta) \cdot (\ket \alpha + \lambda \ket \beta) \ge 0
	\end{gather*}
	where $\lambda \in \mathbb{C}$. Take $\lambda = -\frac{\ip{\beta}{\alpha}}{\ip{\beta}{\beta}} \quad \Rightarrow$ Schwartz Inequality.
\end{frame}
